\chapter{致谢}

在本文的写作过程中,我得到了众多网友的帮助和指点,各位反动学术权威的关心和鼓励。没有你们的帮助,包老师形只影单单枪匹马马不停蹄也难以完成这件超出本人能力的事情。

在此包老师依据我国法律\footnote{《中华人民共和国感谢法》,2010年3月12日。},首先郑重感谢党和政府的栽培,国家和人民的养育,以及有关部门的领导。感谢铁岭 TV,辽宁 TV,将来还有可能感谢 CCTV。

其次将网友们的名单公诸于众,以彰显社会良知、公民勇气。以下排名不分先后,其中多数网友来自水木清华 BBS \TeX 版和 C\TeX 论坛。

\begin{multicols}{2}
\noindent
careworn$@$smth.org\\
Dieken$@$smth.org\\
donated$@$smth.org\\
hkkhhk\\
Hongdong Ji\\
IMB$@$smth.org\\
jjgod$@$smth.org\\
Kov Chai\\
Langpku$@$smth.org\\
LittleLeo$@$smth.org\\
meteorrain$@$smth.org\\
milksea$@$smth.org\\
PaladinHL$@$smth.org\\
PiscesGold$@$smth.org\\
primenumber$@$smth.org\\
snoopyzhao$@$smth.org\\
tex$@$smth.org\\
Xiao Zigang\\
Xubuntu$@$smth.org\\
yakun$@$smth.org\\
yli$@$smth.org\\
yyzz11$@$smth.org\\
张晓南\\
贾朋
\end{multicols}

也借机感谢一下4.80\footnote{166.111.4.80是清华大学校园网早期一个重要的FTP站点。}前站长 Jonny 和水木清华 BBS 前站长 Leeward。十余年前,在我开始浸淫于电脑网络时,这两位高手对我有很大的启发和影响。虽然两位高人淡出公众视野已久,但是他们为人民服务的精神却依然值得我们缅怀与尊敬。

最后还要感谢家人的理解和支持。老妻把她的博士论文给我当作学习 \LaTeX 的试验品;大女儿把她的玉照给我当作插图样板;小女儿把她的名字给我放在献辞页。